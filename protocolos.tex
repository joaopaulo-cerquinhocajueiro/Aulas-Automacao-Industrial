\section{Protocolos Industriais}

Existem vários protocolos de comunicação industrial para diferentes aplicações. Dentre estes, podemos destacar 3 tipos de aplicação mais comuns:
\begin{enumerate}
    \item \label{it:0_1}Comunicação entre CLP e sensores e atuadores simples (nível 1 e nível 0).
    \item \label{it:1_2}Comunicação entre elementos do nível 1 (CLPs, CNCs, inversores, servomotores) e destes com supervisórios (nível 2).
    \item \label{it:2_3}Comunicação entre supervisórios e sistemas de gerência (nível 2 e nível 3).
\end{enumerate}

Redes de comunicação do tipo \ref{it:0_1} normalmente conectam um elevado número de dispositivos e requerem pequenos atrasos na comunicação, embora transmitam uma pequena quantidade de dados. Isto implica em um hardware simplificado, para permitir seu uso numa maior quantidade de dispositivos, e um protocolo determinístico, que permita controlar finamente o momento de aquisição de dados. Isto normalmente é conseguido usando uma camada física e de enlace própria da rede, tais como em redes Modbus RTU, HART, CAN, e AS-i.

Comunicações do tipo \ref{it:1_2} tem as mesmas características de requerer pequenos atrasos, mas acoplada a uma quantidade de dados bem maior, como por exemplo num robô industrial, em que se transmita a posição de cada um de seus eixos. Isto pode ser obtido com uma camada física e de enlace própria, como na Profibus DP, o que é custoso, tanto em termos de desenvolvimento quanto implementação, ou pela aplicação de uma camada ethernet como base da rede, como nas redes Modbus TCP, Profinet e EtherCAT.

Já comunicações do tipo \ref{it:2_3} são normalmente entre computadores e aí o uso delas sobre internet já é bem mais comum, como no caso de OPC-UA.

A tabela \ref{tab:protocolos} lista algumas características destes protocolos de redes industriais, para facilitar a comparação entre eles.

\begin{table}[h]
    \caption{Resumos de protocolos de redes industriais}\label{tab:protocolos}
    \begin{tabular}{l|cp{20mm}p{20mm}p{35mm}}
        \hline
        Protocolo & ano & velocidade & número de pontos & características\\
        \hline
        Modbus RTU & 1979 & $\sim$1~Mbps & 254 & Mestre escravo. Roda sobre RS432 ou RS485\\
        Modbus TCP & - & definido pela rede & ilimitado & Mestre escravo. Roda sobre TCP-IP\\
        HART & 1986 & & 15 & Definido como um sinal digital sobre um sinal 4-20 mA.\\
        CAN & 1987 & & indefinido & Provedor-. Acesso por CSMA-CD.\\
        Profibus DP & 1989 & 12~Mbps & 126 & \\
        AS-i & 1994 & & 62 & Comunicação e alimentação sobre apenas 2 fios.\\
        EtherCAT & 2003 & & & \\
        OPC-UA & 2006 & & & \\
        \hline
    \end{tabular}
\end{table}

\subsection{Modbus}



\subsection{HART}



\subsection{CAN}



\subsection{Profibus DP}



\subsection{AS-i}



\subsection{EtherCAT}



\subsection{OPC-UA}


