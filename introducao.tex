

\section{Início}
Várias máquinas mecânicas de diversas graus de complexidade já foram desenvolvidas ao longo da história, como por exemplo relógios mecânicos desde o século VIII na china e desde o século XIII na Europa e os robôs de Pierre Jaquet-Droz, do século XVIII, ou ainda o mecanismo de Antikythera, datado entre 70 e 50 A.C., e que foi, aparentemente, um relógio astronômico de 27 engrenagens. Mas em termos atuais, podemos considerar o início da automação como sendo a revolução industrial. Em termos mais precisos, podemos usar, de forma levemente arbitrária, a invenção do tear mecânico por Cartwright em 1785 como sendo o ponto inicial da automação industrial. Esta invenção realiza um movimento relativamente complexo de forma automática, usando a energia obtida a partir de uma roda d'água e com a finalidade de gerar um produto (tecido) a partir de uma matéria prima (fio).
\begin{figure}
  \centering
  \includegraphics[width=\textwidth]{cartwright}
  \caption{Tear mecânico de Cartwright. Imagem retirada de...}
\end{figure}

Por volta de 1788 James Watts inventou um mecanismo de regulagem de fluxo de vapor, o que permitiu controlar a potência de caldeiras e outras máquinas a vapor, o que permite utilizar uma potência muito maior que uma roda d'água. Isto foi um grande impulso para a mecanização, com o desenvolvimento de diversos mecanismos e equipamentos. Hoje em dia chamamos isto de Indústria 1.0.

A partir da segunda metade do século XIX, houve a mudança das máquinas a vapor para máquinas a diesel e a energia elétrica. Junto a isto houve a reorganização do espaço das fábricas, fazendo as chamadas linhas de produção. Chamamos este tipo de manufatura de Indústria 2.0.

Neste período, a capacidade de automação ainda ficava muito restrita devido a dificuldade de realizar processos complexos de forma automática. Ou seja, retirava-se grande parte do esforço físico do homem, mas ainda era necessário muito esforço mental. Faltava a invenção do computador.

\begin{figure}
  \centering
  \includegraphics[width=\textwidth]{watt}
  \caption{Mecanismo de regulação de vapor de James Watt. Imagem retirada de...}
\end{figure}

\section{Computação}
Um passo na direção do desenvolvimento do computador se deu em 1820, quando o francês Charles Babbage começou a desenvolver a sua máquina diferencial, voltada para automatizar o processo mental do cálculo matemático, que hoje chamaríamos de uma calculadora mecânica. Ela evoluiu até o conceito da máquina analítica, descrita em 1837, que é considerada o primeiro projeto de computador, embora apenas partes dela tenham sido efetivamente construídas. Estima-se que se fosse totalmente construída, a máquina analítica de Babbage ocuparia um campo de futebol.
\begin{figure}
  \centering
  \includegraphics[width=\textwidth]{babbage}
  \caption{Máquina diferencial de Charles Babbage. Imagem retirada de...}
\end{figure}

Outro passo foi dado por Herman Hollerith em 1880, quando ele usou cartões perfurados como mecanismo de memória de calculadoras mecânicas. Sua intenção era a de automatizar algumas tarefas de tabulação do censo dos EUA. O uso da máquina de Hollerith fez com que o processamento dos dados do censo, que antes duravam 10 anos, fosse concluído em 6.

Em 1936, Alan Turing descreveu um \emph{computador universal} em seu artigo “On Computable Numbers, with an Application to the Entscheidungsproblem”, o que hoje é conhecido como uma \emph{Máquina de Turing}. Suas idéias foram desenvolvidas em 1944, com a construção do Colossus, considerado como o primeiro computador, embora tivesse a função específica de quebrar o código criptográfico alemão na segunda guerra. Ele consistia de um circuito com 1600 a 2400 válvulas com capacidade de processamento de vinte e cinco mil caracteres por segundo. A titulo de comparação, os computadores de uso pessoal de hoje em dia atingem 10 bilhões de cálculos por segundo, enquanto que o mais rápido computador do mundo no início de 2023 é o americano Frontier, que faz $1,194\cdot10^{18}$ cálculos por segundo (\SI{1,194}{exaflops}).

\section{Indústria 3.0}
Ao longo da primeira metade do século XX foram utilizados sistemas eletromecânicos para o controle de processos industriais, chamados circuitos chaveados. Estes sistemas utilizavam chaves e relês (chaves controladas eletricamente) para o controle lógico e para o comando de motores. Um avanço importante nesta área foi dado pelo trabalho de mestrado de Claude Shannon, em 1934, que fez a ligação entre os circuitos chaveados e a lógica matemática, permitindo o uso do ferramental lógico já existente na chamada álgebra de Boole para o projeto de circuitos ainda mais complexos. Veremos mais sobre isto no capítulo \ref{chap:CLP}.
\begin{figure}
  \centering
  \includegraphics[width=\textwidth]{circuitoChaveado}
  \caption{Exemplo de circuito chaveado. Imagem retirada de...}
\end{figure}

Na década de 60 a General Motors fez uma especificação de um equipamento que hoje chamamos de CLP -- Controlador Lógico Programável, que é justamente um computador voltado para o controle de processos industriais, o que simplifica em muito o controle de processos industriais quando comparado com a solução por circuitos chaveados. Em 1968 foi criado o primeiro e eles continuam a ser os principais equipamentos para controlar processos industriais, embora tenham obviamente evoluído muito desde esta versão inicial até os dias de hoje.
\begin{figure}
    \centering
    \includegraphics[width=\textwidth]{clp}
    \caption{Exemplo de CLP.}
\end{figure}

Junto com o CLP, nesta época houve também o desenvolvimento de outros equipamentos, tais como tornos e fresas automatizados (CNCs) e  robôs industriais. Estes equipamentos marcam a chamada Industria 3.0.

Junto com estes sistemas e equipamentos, há o uso cada vez maior de sistemas computacionais de apoio à produção, como sistemas SCADA, PIMS e outros, de controle de estoque e logística.

\section{Indústria 4.0}
O avanço da computação e da eletrônica permite o uso de equipamentos cada vez mais versáteis autônomos. O uso de redes de computadores, sensores baseados em visão computacional, automação dos sistemas de produção, entre outros, leva à Indústria 4.0, bem mais versátil que as anteriores. As redes de dispositivos, com e sem fio, e o barateamento dos sensores permite também a automação de tarefas anexas à própria manufatura, tais como a realização de manutenção preditiva, automatizando a avaliação do estado dos equipamentos utilizados.

\section{Sobre este livro}
Como se vê, a automação industrial moderna embarca muitas áreas, tais como visão computacional, robótica, logística, etc. Além dos próprios processos de manufatura específicos para cada indústria.

Mas podemos simplificar o problema da automação industrial para: fazer processos automáticos para a fabricação de produtos a partir de matérias primas e insumos com o mínimo de interferência e trabalho humano.

Para isto se requer equipamentos que que levantem o estado do processo de fabricação -- sensores -- e com base nisto atuem sobre as matérias primas -- atuadores --. A ligação entre sensores e atuadores deve seguir uma determinada lógica, o que é efetuado de modo também automático por dispositivos controladores.

Além disso, some-se que não existe, pelo menos até agora, automação que seja totalmente independente das pessoas. Logo precisa-se de sistemas supervisórios, para acompanhar o andamento do processo.

Com base nisto, este texto está separado da seguinte forma:
\begin{itemize}
    \item O capítulo \ref{chap:automacao} é mais geral. Define o que é automação industrial, descreve os tipos de automação e apresenta a pirâmide de automação.
    \item O capítulo \ref{chap:instrumentacao} apresenta vários sensores e atuadores usados em processos industriais, bem como algumas definições sobre sensores e atuadores e diagrams do tipo PI\&D.
    \item O capítulo \ref{chap:CLP} apresenta os Controladores Lógicos Programáveis, o principal tipo de controlador usado na indústria.
    \item O capítulo \ref{chap:programacao} apresenta um microcontrolador e faz a ligação entre a programação de um dispositivo como este e a programação de CLPs.
    \item O capítulo \ref{chap:ladder} explica a programação através da linguagem ladder, uma das linguagens padrões de CLPs.
    \item O capítulo \ref{chap:grafcet} detalha a linguagem gráfica grafcet e sua implementação em CLPs: Sequential Function Chart.
    \item O capítulo \ref{chap:scada} apresenta sistemas SCADA - Supervisory Control and Data Acquisition, que é o elemento mais usual para acompanhamento de um processo industrial.
    \item O capítulo \ref{chap:PIMS} sobe mais um degrau na pirâmide de automação e mostra sistemas de gerenciamento de plantas industriais, tais como o PIMS e o MES.
    \item Por último, o caítulo \ref{chap:redes} apresenta redes de computadores e mais especificamente redes industriais.
\end{itemize}
