%!TEX root = principal.tex
\chapter{Programação - arduino}

Como na parte de processamento de um CLP fica um microcontrolador, faz sentido estudar um pouco este elemento para entender a programação de um CLP mais para a frente. Hoje em dia a plataforma mais comum de microcontrolador é o arduino, portanto foi escolhido para exemplificar esta tarefa.

O arduino é uma plataforma de microcontrolador simplificada para permitir facilidade de uso. O nome arduino refere-se a: uma placa com um microcontrolador Atmel, uma linguagem de programação e um ambiente de desenvolvimento para esta linguagem. E também um antigo auto-proclamado rei da Itália, mas este último não importa para nós.

A placa Arduino tem um conector USB para se ligar ao computador. Isto serve tanto para programar o microcontrolador quanto para comunicação entre os 2. Existem várias versões do arduino, pois já que é um sistema \emph{open-source}, quem quiser pode fazer sua versão diferente da placa. Vamos nos referenciar aos arduinos UNO ou outras placas compatíveis com ele.

O arduino UNO tem 4 barras de pinos fêmeas para conexão com outros dispositivos: uma com tensões de alimentação (\bfseries{POWER}), um com 6 entradas analógicas (\bfseries{ANALOG IN}) e 2 com um total de 14 entradas e saídas digitais (\bfseries{DIGITAL}).
\begin{figure}[hbt]
	\centering
	\includegraphics[width=\textwidth]{figuras/Arduino-Uno-Pinout}
	\caption{Pinagem do Arduino Uno}
	\label{fig:pinagem-arduino-uno}
\end{figure}

A linguagem de programação arduino é basicamente a linguagem C++ para microcontroladores ATMEL, mas com algumas funções e definições facilitadoras. A principal diferença entre C++ e a linguagem arduino é que não existe a função main(), mas sim as funções (ou rotinas) \lstinline|setup()| e \lstinline|loop()|. A função \lstinline|setup()| é executada apenas uma vez no momento que o arduino é ligado (ou resetado) e depois o código dentro da função \lstinline|loop()| é executado repetidamente. Com isto o esqueleto de um programa arduino fica:

\begin{lstlisting}[caption=Esqueleto de um programa arduino., label=lst:ardEsquel]
void setup() {
  //codigo a ser executado no inicio
}

void loop() {
  //codigo a ser executado repetidamente
}
 \end{lstlisting}

Lembrando que no arduino, como no C, tudo que tiver depois de \lstinline|//| é comentário e é ignorado pelo compilador.

\section{Piscando um led.}
Vamos passar logo a um exemplo para analisar um programa arduino. As placas de arduino UNO já tem um led ligado ao pino 13, identificado por um \bfseries{L} na placa. Podemos fazer um programa que faça este led piscar.

\begin{lstlisting}[caption= Programa para piscar led.,label=lst:piscaled]
void setup() {
  //codigo a ser executado no inicio
  pinMode(13,OUTPUT); //define o pino 13 como uma saida (led)
}

void loop() {
  //codigo a ser executado repetidamente
  digitalWrite(13,LOW);  //apaga o led
  delay(500);            //espera meio segundo
  digitalWrite(13,HIGH); //acende o led
  delay(500);            //espera meio segundo
}	
\end{lstlisting}

A chamada \lstinline|pinMode(13, OUTPUT);| serve para definir que o pino 13 será uma saída. Obviamente isto só precisa ser feito no início do programa, logo está dentro de \lstinline|setup()|. Se quiséssemos ter uma entrada digital, usaríamos a mesma função, mas trocando OUTPUT por INPUT: \lstinline|pinMode(pino,INPUT);|.

A função que define o valor de um pino digital é a \lstinline|digitalWrite(pino,valor)|. Ela é chamada duas vezes no código \ref{lst:piscaled} dentro de \lstinline|loop()|, uma para apagar o led (gravando \lstinline|LOW|) e outra para acendê-lo (gravando \lstinline|HIGH|). \lstinline|LOW| e \lstinline|HIGH| são duas constantes, de valor 0 e 1, referentes ao zero e um lógico, respectivamente. Em termos físicos, no sistema arduino o \lstinline|LOW| é uma tensão próxima a \SI{0}{V} e o \lstinline|HIGH| uma tensão próxima a \SI{5}{V}.

Um detalhe é que o microcontrolador do arduino funciona numa velocidade de 8 ou \SI{16}{MHz} (dependendo da versão), logo se colocássemos apenas as duas chamadas à função \lstinline|digitalWrite| não veríamos o led piscar, mas teríamos a impressão que ele está aceso com metade da intensidade. Para vermos o led piscar é necessário colocar um atraso, que é justamente obtido pela função \lstinline|delay(x)|, que gera um tempo morto de \lstinline|x| milisegundos.

Se quiséssemos saber o valor de um pino digital que tivesse sido definido como entrada, a função seria \lstinline|digitalRead(pino)|, que retornaria o valor digital naquele pino. Pode-se usar isto por exemplo, para fazer com que uma saída digital seja a cópia de uma entrada digital, como no código \ref{lst:ardCopiaDig}.
\begin{lstlisting}[caption= Programa para acender um led em função de uma entrada digital.,label=lst:ardCopiaDig]
const int pinoSaida = 13;
const int pinoEntrada = 10;

int valor;

void setup() {
  //codigo a ser executado no inicio
  pinMode(pinoSaida,OUTPUT); //define a saida (led)
  pinMode(pinoEntrada,INPUT); //define a entrada
}

void loop() {
  //codigo a ser executado repetidamente
  valor = digitalRead(pinoEntrada); //le a entrada
  digitalWrite(pinoSaida,valor);  //e escreve na saida
}	
\end{lstlisting}

No código \ref{lst:ardCopiaDig} acrescentamos também algumas variáveis. Duas são constantes com os pinos usados. Elas facilitam a leitura do código e também facilitam caso posteriormente quisermos mudar os pinos utilizados. A outra variável armazena o valor lido da entrada, que depois é escrito na saída. 

%Altere o código do seu programa para que ele faça piscar um led ligado ao pino 5 do arduino.
\section{Sinais analógicos}
Em contraste com os sinais digitais, os sinais analógicos são aqueles que podem assumir qualquer valor de tensão. No contexto do arduino, os sinais analógicos estão limitados a valores entre \SI{0}{V} e \SI{5}{V} em relação ao terra.

Para valores analógicos, usamos as funções \lstinline|analogWrite(pino,valor)| e \lstinline|analogRead(pino)|, que, ao contrário das equivalentes digitais, são restritas a alguns pinos específicos. As entradas analógicas são identificadas pelos pinos ANALOG IN (A0 a A5 no arduino) e são ligadas a um conversor analógico/digital (A/D) do microcontrolador, que transforma estes sinais numa palavra binária de 10 bits. Como $2^{10} = 1024$, isto significa que a função \lstinline|analogRead| retorna um valor entre 0 (para uma entrada de 0 V) e 1023 (para uma entrada de 5 V).

Há ainda a função \lstinline|analogWrite|, que gera valores ``analógicos''. Na verdade, o arduino, como a maioria dos microcontroladores, não tem um conversor D/A, logo a função \lstinline|analogWrite| não gera um sinal analógico verdadeiro no pino. O que esta função faz é gerar um sinal modulado por largura de pulso - PWM (\emph{Pulse Width Modulation}).

O sinal PWM é um trem de pulsos digital, com frequência da ordem de \SI{500}{Hz} (no caso do arduino) cuja razão entre o tempo em alto e o perído (conhecida como \emph{duty cycle}) pode ser alterada pelo parâmetro passado, como mostra a figura \ref{fig:pwm}.
\begin{figure}[hbt]
	\centering
	\includegraphics[width=\textwidth]{figuras/pwm}
	\caption{Reta modulada em largura de pulso (PWM).}
	\label{fig:pwm}
\end{figure}	
 Se um sinal PWM é enviado a um pino com um led, ele piscará 500 vezes por segundo, o que é muito rápido para o olho humano, de modo que na prática o que se vê quando se varia o duty cycle de um sinal PWM que aciona um led é uma variação de sua intensidade. Logo um sinal PWM funciona, para muitas aplicações, como um sinal analógico.

Tal como apenas alguns pinos tem a capacidade de ler sinais analógicos, também não são todos os pinos do arduino que conseguem gerar um sinal PWM, logo a função \lstinline|analogWrite| está restrita aos pinos 3, 5, 6, 9, 10 e 11 (do arduino UNO. Outros podem usar outros pinos). Um outro detalhe que vale a pena ressaltar é que enquanto a função \lstinline|analogRead| gera um valor entre 0  1023, a \lstinline|analogWrite| recebe como parâmetro um valor entre 0 e 255 apenas.

Uma função útil do arduino para lidar com este tipo de situação é a função \lstinline|map(valor, minIn, maxIn, minOut, maxOut)|, que faz uma transformação linear de valor de acordo com a seguinte equação:
\begin{equation*}
(\text{valor} - \text{minIn}) \times \frac{\text{maxOut} - \text{minOut}}{\text{maxIn} - \text{minIn}} + \text{minOut}
\end{equation*}

A partir desta função, um código que leia o valor gerado pelo potenciômetro (em A0) e controle a intensidade do led no pino 5 poderia ser simplesmente:
\begin{lstlisting}
analogWrite(5,map(analogRead(A0),0,1024,0,256));
\end{lstlisting}

Note que o valor lido pelo \lstinline|analogRead| não precisa ser usado apenas na função \lstinline|analogWrite| mas pode ser usado para outra finalidade, como por exemplo alterar um atraso.
%Faça um programa que altere a frequência com que um led pisca em função da posição de um potênciometro.
%Repita o problema anterior, só que agora fazendo com que o led RGB ligado nos pinos 9, 10 e 11 pisque na sequencia vermelho, verde e azul, com a frequência definida pelo potenciômetro.
  
\section{Controle: for e if}
Até aqui foram feitos programas puramente sequenciais, porém em vários momentos é interessante realizar operações repetidas vezes ou realizar algumas tarefas apenas em situações específicas. Para estes casos existem comandos como o \lstinline|for| e o \lstinline|if|.
O comando for serve para tarefas repetidas. Por exemplo, se quiséssemos inicializar os pinos de 2 a 10 como saídas com valor 0, poderíamos usar o seguinte código:
\begin{lstlisting}
for(int i = 2; i<11; i++){
	pinMode(i,OUTPUT);
	digitalWrite(i,LOW);
}	
\end{lstlisting}
O que este código faz é definir uma variável local \lstinline|i| com valor inicial 2 depois ele checa se \lstinline|i| é menor que 11 e, se for, executa os comandos que estão entre as chaves ``\{'' e ``\}'', incrementa a variável \lstinline|i| (\lstinline|i++|) e checa novamente. Quando \lstinline|i < 11| for falso, ele sai do laço.

%Faça um programa que faça o led piscar suavemente (a intensidade variando entre apagado e totalmente aceso)
O comando \lstinline|if| executa um determinado código apenas se determinada condição for verdadeira. Por exemplo, para acender um led apenas se um sinal analógico for maior que a metade da escala (512 = 1024/2)  pode-se escrever:
\begin{lstlisting}
if(analogRead(A0) >= 512){
	digitalWrite(13,HIGH);
}
\end{lstlisting}

Note que este código apenas fará alguma coisa se a condição for verdadeira. Para fazer uma coisa OU outra usa-se o comando \lstinline|else| após o \lstinline|if|:
\begin{lstlisting}
if(analogRead(A0) >= 512){
	digitalWrite(13,HIGH);
} else {
	digitalWrite(13,LOW);
}
\end{lstlisting}
%Faça o led rgb piscar suavemente em vermelho, azul ou verde dependendo do potenciômetro.
\section{Comunicação com o computador}
Como já dito, a conexão USB do arduino serve também para a comunicação do mesmo com o computador. Do lado do computador o arduino aparece como uma porta serial e o próprio programa contém um terminal serial pelo qual é possível se comunicar com o arduino. Do lado do arduino, os pinos 0 e 1 são os pinos transmissor e receptor ligados ao USB .

A programação é feita através do objeto  \lstinline|Serial|. Este objeto tem vários comandos, porém para nós os que interessam neste momento são:
\begin{description}
	\item[Serial.begin(baud)] Inicializa a comunicação serial na velocidade (baud rate) indicada.
	\item[Serial.read()] Retorna o valor de um byte recebido ou -1 caso não tenha sido recebido nenhum byte.
	\item[Serial.print(dado)] Se dado for um char ou uma string (texto), envia dado. Se dado for um número, envia este número como uma string. 
	\item[Serial.println(dado)] Igual ao \lstinline|println|, mas é acrescentada uma quebra de linha após dado.
	\item[Serial.available()] Retorna o número de bytes recebidos que ainda não foram lidos.
\end{description}

%Controle a intensidade das cores do led rgb pelo computador, enviando “cn” pela serial, onde 'c' é igual a 'r', 'g' ou 'b' e 'n' é um byte.
