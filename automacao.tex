%!TEX root = principal.tex
A palavra automação vem do latim automatus -- mover por si mesmo. Logo a automação de uma tarefa consiste em fazer esta tarefa ser realizada sem trabalho humano. Isto pode ser por diversos motivos: seja por que é uma tarefa perigosa e portanto queremos aumentar a segurança das pessoas, como num processo que envolva alta temperatura, por exemplo; seja para fazer a tarefa de forma mais rápida, seja para melhorar a qualidade do produto final ou seja porque simplesmente o custo do trabalho humano é muito elevado. Logo, podemos definir automação da seguinte forma:
\begin{quote}
  Automação é a substituição do trabalho humano para melhorar segurança, qualidade, produção e custos.
\end{quote}

Neste contexto, automação industrial é nada mais que a automação de um sistema industrial, ou de um sistema de manufatura. Embora manufatura venha de fazer com as mãos, a revolução industrial mudou seu conceito, passando a significar a fabricação de praticamente qualquer produto. Do ponto de vista econômico, a manufatura é a transformação de materiais (matéria prima) em itens de maior valor (produto). Isto é conseguido por uma determinada sequência de processos químicos e físicos. De forma mais sucinta:
\begin{quote}
	Manufatura é a transformação de matéria prima em produtos pela aplicação de um ou mais processos.
\end{quote}

Logo, a automação industrial consiste em fazer os processos necessários para a manufatura com o mínimo de esforço ou interferência humana, visando melhor segurança, qualidade, produção e custo. Note que por esforço humano, queremos dizer tanto esforço físico quanto mental, logo uma máquina bastante complexa mas que funcione a manivela, não se classificaria como automação; da mesma forma uma máquina que não exija esforço físico mas requer atenção constante também não é automatizada (seria apenas mecanizada).

\section{História}
É interessante pegar alguns pontos chaves na história da automação. Embora várias máquinas mecânicas de diversas graus de complexidade já existissem, como por exemplo relógios mecânicos desde o século VIII na china e desde o século XIII na Europa e os fantásticos robôs de Pierre Jaquet-Droz, do século XVIII, considera-se que a revolução industrial iniciou com a invenção do tear mecânico por Cartwright em 1785, que realiza um movimento relativamente complexo de forma automática a partir de uma roda d'água.

Um outro grande avanço ocorreu por volta de 1788, com a invenção do mecanismo de regulagem de fluxo de vapor de James Watt, o que permitia então controlar a potência de caldeiras e outras máquinas a vapor, controlando uma energia muito maior que uma roda d'água. Isto foi um grande impulso para a mecanização, mas a verdadeira automatização ainda ficava muito restrita devido a dificuldade de realizar processos complexos de forma automática. Ou seja, retirava-se grande parte do esforço físico do homem, mas ainda era necessário muito esforço mental.

Em 1820 Babbage começou a desenvolver a sua máquina diferencial, que hoje chamaríamos de uma calculadora mecânica. Ela evoluiu até o conceito da máquina analítica, descrita em 1837, que é considerada o primeiro projeto de computador, embora apenas partes dela tenham sido efetivamente construídas.

 Em 1880, Herman Hollerith criou um novo método baseado na
utilização de cartões perfurados, para automatizar algumas tarefas de
tabulação do censo dos EUA que antes duravam 10 anos. Com o
método, o processo era concluído em 6.

Ao longo da primeira metade do século XX foram utilizados muitos sistemas eletromecânicos para o controle de processos industriais. Eram os chamados circuitos chaveados, que utilizavam relês para o controle lógico e para o comando de motores.

Em 1936, Alan Turing descreveu um \emph{computador universal} em seu artigo “On Computable Numbers, with an Application to the Entscheidungsproblem”, o que hoje é conhecido como uma Máquina de Turing. Suas idéias foram desenvolvidas em 1944, com a construção do Colossus, considerado como o primeiro computador, embora tivesse a função específica de quebrar o código criptográfico alemão na segunda guerra. Ele consistia de um circuito com 1600-2400 válvulas com capacidade de processamento de 25k caracteres/s. A titulo de comparação, os computadores de casa de hoje em dia atingem 10 bilhões de cálculos por
segundo.

O avanço da eletrônica fez que que a capacidade de processamento dos computadores aumentasse de forma exponencial. Atualmente o mais rápido computador do mundo é o chinês Tianhe-2, que faz 33,86 quatrilhões de cálculos por segundo, consumindo 24MW.

Na década de 60 a General Motors fez uma especificação de um CLP -- Controlador Lógico Programável, que é um computador voltado para o controle de processos industriais. Em 1968 foi criado o primeiro.

Hoje em dia toda automação está relacionada a sistemas computadorizados, seja em CLPs, CNC, robôs industriais, automação dos sistemas de apoio a produção, entre outros.

\section{Classificação}

Hoje em dia se definem, grosso modo, 3 tipos de automação, tal qual mostra a figura \ref{fig:tipos_automacao}: fixa, flexível e programável.
\begin{figure}[hbt]
	\begin{center}
%		\includegraphics[width=0.6\textwidth]{tipos_automacao}
	\tikzstyle{area}=[draw,rectangle,rounded corners,text centered,text width=2.2cm,minimum height=1.2cm]
	\begin{tikzpicture}
		\draw[very thick, ->, >=stealth'](0,0)--(5.5,0);
		\draw(5.5,0)node[below left]{Diversidade};
		\draw[very thick, ->, >=stealth'](0,0)--(0,3.3);
		\draw(0,3.3)node[above left, rotate=90]{Quantidade};
		\draw(1.25,2.5)node[area]{Fixa} (2.5,1.55)node[area]{Flexível} (4,0.6)node[area]{Programável};
	\end{tikzpicture}
	\end{center}
	\caption{Tipos de automação industrial quanto a quantidade e diversidade de produtos.}
	\label{fig:tipos_automacao}
\end{figure}

A automação fixa é aplicada à produção de um único produto (ou com mínimas variações), em grandes quantidades: refinaria de petróleo, parafuso, tampas de garrafa, clips, biscoito, cerveja, etc. Ela utiliza equipamentos específicos para aquela tarefa, que portanto tem alto custo mas grande produtividade.

A automação flexível é aplicada à produção de produtos parecidos, em que pequenas modificações permitem a alteração do produto, como por exemplo mudança de um perfil a ser prensado ou extrudado ou a mudança das quantidades do mesmo conjunto de matérias primas (mudança de receita). Tipicamente é feita a chamada fabricação em lotes, onde entre um lote e outro se alteram as peças e/ou as sequências a serem seguidas de forma automática para ter o menor tempo parado possível. Exemplos são livros, circuitos integrados, potes de plástico, máquinas de café.

A automação programável é para produção de produtos diferentes mas cujo volume de produção não justifica um processo único. Ela usa máquinas de propósito geral, tais como robôs, ferramentas de controle numérico e impressoras 3d, onde a definição do processo é quase toda feita por \emph{software}, de modo que o custo do maquinário é diluído em diversos produtos.

A tendência é cada vez mais ter a automação flexível e programável aumentando a capacidade de produção, de modo que a flexível vai ocupando nichos da fixa e a programável da flexível. Apesar disso, em vários casos é difícil imaginar alguns produtos deixando de utilizar a automação fixa.

\section{Pirâmide de automação}

A automação em larga escala de uma grande indústria ou de um conjunto de indústrias é mais complexa que a manufatura: envolve problemas de abastecimento, armazenagem, análise de mercado, exigências ambientais, entre várias outras coisas. Uma forma de se separar os diferentes problemas da automação é através da chamada Pirâmide de Automação, mostrada na figura \ref{fig:piramide}.

\begin{figure}[htb]
	\begin{center}
\begin{tikzpicture}[y=0.80pt, x=0.8pt,xscale=0.4,yscale=-0.4, inner sep=0pt, outer sep=0pt]
    \path[fill=black] (530,582.41803) node[above right] (text3018)
      {Instrumentação};
    \path[fill=black] (530,511.83002) node[above right] (text3022) {Controle};
    \path[fill=black] (530,441.0715) node[above right] (text3794)
      {Supervisão};
    \path[fill=black] (530,370.13507) node[above right] (text3798) {Gerência
      de manufatura};
    \path[fill=black] (530,299.51303) node[above right] (text3804)
      {Planejamento estratégico};
      \path[cm={{1.01932,0.0,0.0,1.01932,(-6.1462,-4.25386)}},draw=black,fill=c00ffff,miter
        limit=4.00,line width=2pt] (320.0000,252.3622) -- (423.9230,432.3622) --
        (527.8461,612.3622) -- (320.0000,612.3622) -- (112.1539,612.3622) --
        (216.0770,432.3622) -- cycle;
      \path[cm={{0.80176,0.0,0.0,0.79778,(63.47204,51.63306)}},draw=black,fill=c00ff00,miter
        limit=4.00,line width=2pt] (320.0000,252.3622) -- (423.9230,432.3622) --
        (527.8461,612.3622) -- (320.0000,612.3622) -- (112.1539,612.3622) --
        (216.0770,432.3622) -- cycle;
      \path[cm={{0.60132,0.0,0.0,0.59833,(127.61319,101.96534)}},draw=black,fill=cffff00,miter
        limit=4.00,line width=2pt] (320.0000,252.3622) -- (423.9230,432.3622) --
        (527.8461,612.3622) -- (320.0000,612.3622) -- (112.1539,612.3622) --
        (216.0770,432.3622) -- cycle;
      \path[cm={{0.40088,0.0,0.0,0.39889,(191.75434,152.29762)}},draw=black,fill=cff6600,miter
        limit=4.00,line width=2pt] (320.0000,252.3622) -- (423.9230,432.3622) --
        (527.8461,612.3622) -- (320.0000,612.3622) -- (112.1539,612.3622) --
        (216.0770,432.3622) -- cycle;
      \path[cm={{0.20044,0.0,0.0,0.19944,(255.8955,202.62989)}},draw=black,fill=cff5555,miter
        limit=4.00,line width=2pt] (320.0000,252.3622) -- (423.9230,432.3622) --
        (527.8461,612.3622) -- (320.0000,612.3622) -- (112.1539,612.3622) --
        (216.0770,432.3622) -- cycle;
    \begin{scope}[shift={(3.29592,0)}]
      \path[fill=black] (309.04346,582.41803) node[above right] (text3824) {0};
      \path[fill=black] (310.00049,511.8302) node[above right] (text3828) {1};
      \path[fill=black] (309.28271,441.0715) node[above right] (text3832) {2};
      \path[fill=black] (309.00244,370.13507) node[above right] (text3836) {3};
      \path[fill=black] (309.45361,299.51303) node[above right] (text3840) {4};
    \end{scope}
    \path[fill=black] (79.834969,582.41803) node[above left] (text3858) {Sensores
      eatuadores};
    \path[fill=black] (79.875984,511.8302) node[above left] (text3862) {CLP, SDCD,
      CNC};
    \path[fill=black] (81.256844,441.0715) node[above left] (text3866) {SCADA,
      HMI};
    \path[fill=black] (80.245125,370.13507) node[above left] (text3870) {PIMS, MES,
      LIMS, WMS};
    \path[fill=black] (79.999031,299.51303) node[above left] (text3874) {ERP};

\end{tikzpicture}
	\end{center}
	\caption{Pirâmide de automação.}
	\label{fig:piramide}
\end{figure}

Note que esta não é a única representação da pirâmide: umas começam pelo 1, outras tem apenas 4 camadas, e assim por diante, logo mais importante que o número de cada camada é o que tais camadas significam.

\begin{description}
	\item[Nível 0: Instrumentação] Camada onde se encontram instrumentos, sensores, motores,
máquinas, etc. Consistem nos equipamentos do chamado ``\emph{chão de fábrica}''.
\item[Nível 1: Controladores] Controle automático da planta -- onde se localizam os Controladores
Lógico-Programáveis (CLP), os Sistemas Digitais de Controle Distribuído (SDCD), os Controles Numéricos Computadorizados (CNC) e/ou computadores de controle.
\item[Nível 2: Supervisão] Supervisão e controle do processo através de Interfaces Homem-Má
quina (IHMs) ou SCADA (\emph{Supervisory Control And Data Acquisition}).
\item[Nível 3: Gerenciamento da Manufatura] Gestão dos recursos da planta e controle da produção.  Sistemas PIMS
(\emph{Process Information Management System}) e MES (\emph{Manufacturing Execution Systems}).
\item[Nível 4: Gerenciamento da Empresa] Gestão dos recursos e produção da empresa como um todo. ERP –
\emph{Enterprise Resources Planning}.
\end{description}

Este texto faz um estudo da automação industrial de modo \emph{top-down}: começando do nível 3 até o nível 0. O nível 4 é mais importante para um estudo de engenharia de processo e portanto não será abordado.